\documentclass[11pt,a4paper]{article}
\usepackage[utf8]{inputenc}
\usepackage{amsmath}
\usepackage{amsfonts}
\usepackage{amssymb}
\usepackage{geometry}
\usepackage{listings}
\usepackage{xcolor}
\usepackage{hyperref}

\geometry{margin=1in}

\definecolor{codegreen}{rgb}{0,0.6,0}
\definecolor{codegray}{rgb}{0.5,0.5,0.5}
\definecolor{codepurple}{rgb}{0.58,0,0.82}
\definecolor{backcolour}{rgb}{0.95,0.95,0.92}

\lstdefinestyle{mystyle}{
    backgroundcolor=\color{backcolour},   
    commentstyle=\color{codegreen},
    keywordstyle=\color{magenta},
    numberstyle=\tiny\color{codegray},
    stringstyle=\color{codepurple},
    basicstyle=\ttfamily\footnotesize,
    breaklines=true,
    captionpos=b,                    
    keepspaces=true,                 
    showspaces=false,                
    showstringspaces=false,
    showtabs=false,                  
    tabsize=2
}

\lstset{style=mystyle}

\title{CTT Topological Resonator: Core Integration Specifications}
\author{Americo Simoes}
\date{February 2026}

\begin{document}

\maketitle

\section{Introduction}
The CTT Topological Resonator is designed as a high-precision phase-tracking module for the SatDump manifold. It provides frequency stabilization for L-band signals, specifically targeting the $1.7$~GHz downlink window.

\section{Mathematical Framework}
The core of the resonator utilizes a Discrete-Time Topological Recurrence. Given a complex IQ input stream $z[n]$, the phase state $\phi$ evolves according to the following control law:

\[ \phi[n+1] = \phi[n] + \omega + K \cdot \text{Im}\{z[n] \cdot e^{-j\phi[n]}\} \]

Where:
\begin{itemize}
    \item $\phi[n]$ is the instantaneous phase at sample $n$.
    \item $\omega$ is the center frequency offset ($2\pi f_c / f_s$).
    \item $K$ is the topological coupling constant.
\end{itemize}

\section{Software Implementation}
To prevent linker optimization errors (dead code elimination), the module is hard-linked into the SatDump core.

\subsection{Registry Injection}
The module is registered in \texttt{src-core/pipeline/module.cpp} using the following macro call:
\begin{lstlisting}[language=C++]
REGISTER_MODULE(CTTTopologicalResonator);
\end{lstlisting}

\subsection{Module Identification}
The module manifests in the CLI with the following metadata:
\begin{itemize}
    \item \textbf{Internal ID:} \texttt{ctt\_phi24\_resonator}
    \item \textbf{Type:} \texttt{ProcessingModule}
\end{itemize}

\section{Compilation and Verification}
The binary was successfully compiled on Fedora using \texttt{GCC 15}. Verification was performed via:
\begin{lstlisting}[language=bash]
./satdump modules list | grep ctt
\end{lstlisting}

\end{document}
